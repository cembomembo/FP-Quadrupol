\documentclass[a4paper]{article}

% --- Page layout and spacing ---
\usepackage[top=3cm, left=3.5cm, right=3.5cm, bottom=3cm]{geometry}
\usepackage[utf8]{inputenc}      % input encoding
\usepackage[T1]{fontenc}         % font encoding
\usepackage[english]{babel}
\usepackage{setspace}
\setlength{\parindent}{0pt}      % paragraph indentation
\setlength{\parskip}{0.8em}      % space between paragraphs
\setstretch{1.2}                 % line spacing
\usepackage{tocloft}             % section spacing in ToC
\setlength{\cftbeforesecskip}{10pt}
\setlength{\cftbeforesubsecskip}{4pt}
\usepackage{titlesec}            % section title spacing
\titlespacing*{\section}{0pt}{5.0ex plus 1ex minus .2ex}{1.0ex plus .2ex}
\titlespacing*{\subsection}{0pt}{3.0ex plus .5ex minus .2ex}{0.8ex plus .2ex}
\titlespacing*{\subsubsection}{0pt}{2.0ex plus .5ex minus .2ex}{0.8ex plus .2ex}

% --- Math and symbols ---
\usepackage{amsmath, amssymb}    % standard math
\usepackage{empheq}              % boxed equations etc.
\DeclareMathOperator{\artanh}{artanh}
\DeclareMathOperator{\sgn}{sgn}
\usepackage{bm}                  % bold math symbols
\usepackage{cancel}              % strikeout in math
\usepackage{siunitx}             % proper units
\renewcommand{\arraystretch}{0.7}

% --- Graphics and floats ---
\usepackage{graphicx}
\usepackage{float}
\usepackage{wrapfig}
\usepackage[justification=centering]{caption}
\usepackage{subcaption}
\captionsetup[figure]{font=small}

% --- Layout helpers ---
\usepackage{boxedminipage}
\usepackage{enumitem}
\usepackage{afterpage}
\usepackage{changepage}
\usepackage{pdfpages}           % include external PDFs
\usepackage{esvect}             % nice vector arrows
\usepackage{hyperref}           % hyperlinks

% --- Bibliography setup ---
\usepackage{csquotes}
\usepackage[backend=biber,style=numeric,sorting=none]{biblatex}
\addbibresource{references.bib}

% --- Fonts ---
\usepackage{lmodern}            % Computer Modern look across TeX distros


% --- Title page ---
\title{\textbf{Quadrupol Mass Spectrometer}}
\author{
  \\Preparation Report \\\\\\\\\\\\
  \textbf{Cem Boyaci} \\
  cemb93@zedat.fu-berlin.de \\\\\\
  \textbf{Javier Bellido Roldán}\\
  bellidoroj98@zedat.fu-berlin.de \\\\\\
  \textbf{Leon Goldammer} \\
  lg4278fu@zedat.fu-berlin.de \\\\
}
\date{}

\begin{document}
\maketitle
\thispagestyle{empty}

\section*{}
\begin{center}
\vspace{3cm}
Tutor: Kati Hubmann \\[1cm]
\textbf{Fortgeschrittenenpraktikum, WS 2025/2026}\\
Berlin, 12.01.2026\\
Freie Universität Berlin\\
Fachbereich Physik
\end{center}


% --- Table of contents ---
\clearpage
\renewcommand*\contentsname{\huge Contents}
{
  \pagenumbering{gobble}
  \tableofcontents
  \clearpage
}
\pagenumbering{arabic}


% --- Introduction ---
\newpage
\setcounter{page}{1}

\section{Introduction}

Mass spectrometry is a versatile analytical technique used to investigate the composition of atomic and molecular species and plays an important role in many areas of modern science.
In physics and chemistry, it is widely applied whenever gas mixtures or residual gases need to be analyzed with high sensitivity and selectivity.
Among the various types of mass spectrometers, the quadrupole mass spectrometer stands out due to its compact design, robustness, and suitability for operation in high and ultra-high vacuum environments.
For this reason, it has become a standard tool in vacuum diagnostics, residual gas analysis, and experimental research setups~\cite{jousten_ellefson_partialdruckmessung_2017}.

In this experiment, gas samples are ionized, mass-selected by a quadrupole mass filter, and detected as an electrical signal, resulting in characteristic mass spectra.
Different gases and gas mixtures are examined, allowing their composition and fragmentation patterns to be analyzed.
In addition, the influence of experimental parameters such as applied voltages, pressure conditions, and operating modes on mass resolution, transmission, and signal quality is investigated.
The measurements provide a practical basis for the interpretation of mass spectra and the assessment of the performance limits of a quadrupole mass spectrometer.


% --- Physical Principles ---
\section{Physical Principles}

In this section, the physical principles underlying the operation of a quadrupole mass spectrometer are outlined.
The discussion follows the path of the particles through the instrument, from ionization and transport under vacuum conditions to mass-selective filtering, detection, and the interpretation of the resulting mass spectra.

\subsection{Mass Spectrometry and Ionization}

Mass spectrometry is an analytical technique used to investigate the composition of atomic and molecular species by separating charged particles according to their mass-to-charge ratio.
Since neutral atoms and molecules cannot be manipulated by electric fields, the species under investigation must first be converted into ions.
The measured quantity in a mass spectrometer is therefore not the mass itself, but the ratio of mass to charge $m/z$, where $m$ denotes the ion mass and $z$ its charge state in units of the elementary charge.
Different ions with the same mass but different charge states are thus separated at different $m/z$ values, while ions of different mass may appear at similar $m/z$ positions if their charge states differ.

A mass spectrometer generally consists of three essential components: an ion source, a mass-selective analyzer, and an ion detector.
In the ion source, neutral gas particles are ionized and accelerated into the analyzer region.
The analyzer separates the ions according to their $m/z$ value using electric or magnetic fields, and the transmitted ions are finally detected and converted into an electrical signal.
A schematic overview of these basic components is shown in Fig.~\ref{fig:mass_spectrometer}.
In the present experiment, mass selection is performed by a quadrupole mass filter (QMF) operating with time-dependent electric fields~\cite{QMSAnleitung}.

\begin{figure}[H]
  \centering
  \includegraphics[width=0.9\linewidth]{../resources/figures/mass_spectrometer.png}
  \caption{Schematic overview of the main components of a mass spectrometer, consisting of an ionization source, a mass analyzer, and an ion detector, operated under vacuum conditions~\cite{siuzdak_ms_ionization}.}
  \label{fig:mass_spectrometer}
\end{figure}

Ionization in the quadrupole mass spectrometer used in this experiment is achieved by electron impact ionization.
Electrons are emitted from a heated filament by thermionic emission and accelerated to energies typically on the order of several tens of electronvolts.
When these electrons collide with neutral gas molecules, they can remove an electron and thereby produce positive ions.
Due to the relatively high electron energies, this process often transfers excess energy to the molecule, leading to the formation of fragment ions in addition to the molecular parent ion.

As a consequence, a measured mass spectrum generally consists of multiple peaks corresponding to different ionic fragments of the same molecule.
These fragmentation patterns are characteristic for a given species and can be used to identify unknown gases or gas mixtures.
Under typical operating conditions of electron impact ionization, singly charged ions ($z = 1$) dominate, so that the measured mass-to-charge ratio $m/z$ closely corresponds to the molecular or fragment mass $m$.
Multiply charged ions may occur but are significantly less abundant and usually play a minor role in residual gas analysis \cite{jousten_ellefson_partialdruckmessung_2017}.

\subsection{Vacuum Requirements and Mean Free Path}

\subsection{Principle of the Quadrupole Mass Filter}

\subsection{Ion Motion in the Quadrupole Field}

\subsection{Stability Diagram and Mass Selection}

\subsection{Detection and Mass Spectra}


% --- Experimental Setup ---
\section{Experimental Setup}

xxx

\subsection{xxx}

xxx


% --- Procedure ---
\section{Procedure}

\subsection{xxx}

xxx


% --- References ---
\newpage
\setstretch{1.0}
\printbibliography[heading=bibintoc]

\section*{Author's Note}
AI-based writing and programming tools were used in a supporting role to refine the wording of this report and to assist in formatting Python and LaTeX code.
All scientific analysis, data evaluation, and interpretation were carried out independently by the authors.


\end{document}
